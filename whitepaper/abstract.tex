Today's leading cryptocurrencies, Bitcoin and Ethereum, as well as myriad
others, are built on a public consensus network based on proof-of-work (PoW)
mining. In this work, we construct a new primitive called
Non-Interactive-Proofs-of-Proof-of-Work (NIPoPoWs) that can be adapted into
existing PoW-based cryptocurrencies to improve their performance and extend
their functionality. Unlike a traditional blockchain client which must verify
the entire linearly-growing chain of PoWs, clients based on NIPoPoWs require
resources only logarithmic in the length of the blockchain.

Our work  extends and improves prior work that introduced ``proofs of proofs of
work''. Compared to theirs, our protocol is non-interactive, requiring only one
round of communication thus resolving the main open question that was left open.
We also identify and correct a double-spend vulnerability and security flaw in
that work. We also extend their proofs to be able to prove general useful
predicates about facts buried deep within a blockchain, an extension necessary
for the basic application of accepting payments.

We provide empirical validation for NIPoPoWs through an implementation and
benchmark study, in the context of two new applications: First, we consider a
multi-client blockchain that supports all proof-of-work currencies rather than
just one, with up to 90\% reduction in bandwidth.  Second, we discuss a
``cross-chain ICO'' application that spans multiple independent blockchains.
