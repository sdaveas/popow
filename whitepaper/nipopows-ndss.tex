\documentclass[conference]{IEEEtran}

\newif\iftwocolumn
\newif\ifanonymous
\newif\ifonecolumn
\newif\iflncs
\newif\ifieee
\newif\ifhasappendix
\newif\ifndss
\twocolumntrue
\anonymoustrue
\ieeetrue
\ndsstrue

\usepackage{preamble}
\pagestyle{plain}

\usepackage{preamble}

\begin{document}
%
% paper title
% can use linebreaks \\ within to get better formatting as desired
\import{./}{title.tex}


% author names and affiliations
% use a multiple column layout for up to three different
% affiliations
% \author{\IEEEauthorblockN{Michael Shell}
% \IEEEauthorblockA{Georgia Institute of Technology\\
% someemail@somedomain.com}
% \and
% \IEEEauthorblockN{Homer Simpson}
% \IEEEauthorblockA{Twentieth Century Fox\\
% homer@thesimpsons.com}
% \and
% \IEEEauthorblockN{James Kirk\\ and Montgomery Scott}
% \IEEEauthorblockA{Starfleet Academy\\
% someemail@somedomain.com}}

% conference papers do not typically use \thanks and this command
% is locked out in conference mode. If really needed, such as for
% the acknowledgment of grants, issue a \IEEEoverridecommandlockouts
% after \documentclass

% for over three affiliations, or if they all won't fit within the width
% of the page, use this alternative format:
%
%\author{\IEEEauthorblockN{Michael Shell\IEEEauthorrefmark{1},
%Homer Simpson\IEEEauthorrefmark{2},
%James Kirk\IEEEauthorrefmark{3},
%Montgomery Scott\IEEEauthorrefmark{3} and
%Eldon Tyrell\IEEEauthorrefmark{4}}
%\IEEEauthorblockA{\IEEEauthorrefmark{1}School of Electrical and Computer Engineering\\
%Georgia Institute of Technology,
%Atlanta, Georgia 30332--0250\\ Email: see http://www.michaelshell.org/contact.html}
%\IEEEauthorblockA{\IEEEauthorrefmark{2}Twentieth Century Fox, Springfield, USA\\
%Email: homer@thesimpsons.com}
%\IEEEauthorblockA{\IEEEauthorrefmark{3}Starfleet Academy, San Francisco, California 96678-2391\\
%Telephone: (800) 555--1212, Fax: (888) 555--1212}
%\IEEEauthorblockA{\IEEEauthorrefmark{4}Tyrell Inc., 123 Replicant Street, Los Angeles, California 90210--4321}}




% use for special paper notices
%\IEEEspecialpapernotice{(Invited Paper)}



\IEEEoverridecommandlockouts
\makeatletter\def\@IEEEpubidpullup{9\baselineskip}\makeatother
\IEEEpubid{\parbox{\columnwidth}{
    Network and Distributed Systems Security (NDSS) Symposium 2019\\
    24-27 February 2019, San Diego, CA, USA\\
    ISBN x-xxxxxxx-xx-x\\
    http://dx.doi.org/xx.xxxxx/ndss.2019.xxxxx\\
    www.ndss-symposium.org
}
\hspace{\columnsep}\makebox[\columnwidth]{}}


% make the title area
\maketitle



\begin{abstract}
%\boldmath
Open consensus protocols based on proof-of-work (PoW) mining are at the core of
cryptocurrencies such Bitcoin and Ethereum, as well as many others. In this
work, we construct a new primitive called
Non-Interactive-Proofs-of-Proof-of-Work (NIPoPoWs) that can be adapted into
existing PoW-based cryptocurrencies to improve their performance and extend
their functionality. Unlike a traditional blockchain client which must verify
the entire linearly-growing chain of PoWs, clients based on NIPoPoWs require
resources only logarithmic in the length of the blockchain. NIPoPoWs are thus
succinct proofs and require only a single message between the prover and the
verifier of the transaction.

With our construction we are able to prove a broad array of useful predicates in
the context of cross PoW-based blockchain transfers of assets,  including
predicates about facts buried deep within a blockchain which is necessary for
the basic application of accepting payments.

We provide empirical validation for NIPoPoWs through an implementation and
benchmark study, in the context of two new applications: First, we consider a
multi-client blockchain that supports all proof-of-work currencies rather than
just one, with up to 90\% reduction in bandwidth.  Second, we discuss a
``cross-chain ICO'' application that spans multiple independent blockchains.
Using our experimental data, we provide concrete parameters for our scheme.

\end{abstract}
% IEEEtran.cls defaults to using nonbold math in the Abstract.
% This preserves the distinction between vectors and scalars. However,
% if the conference you are submitting to favors bold math in the abstract,
% then you can use LaTeX's standard command \boldmath at the very start
% of the abstract to achieve this. Many IEEE journals/conferences frown on
% math in the abstract anyway.

% no keywords




% For peer review papers, you can put extra information on the cover
% page as needed:
% \ifCLASSOPTIONpeerreview
% \begin{center} \bfseries EDICS Category: 3-BBND \end{center}
% \fi
%
% For peerreview papers, this IEEEtran command inserts a page break and
% creates the second title. It will be ignored for other modes.
%%\IEEEpeerreviewmaketitle



% \section{Introduction}
% % no \IEEEPARstart
% This demo file is intended to serve as a ``starter file''
% for IEEE conference papers produced under \LaTeX\ using
% NDSStran.cls, which is based on IEEEtran.cls version 1.7 and later.
% % You must have at least 2 lines in the paragraph with the drop letter
% % (should never be an issue)
% I wish you the best of success.
%
% \hfill mds
%
% \hfill January 11, 2007
%
% \subsection{Subsection Heading Here}
% Subsection text here.
%
%
% \subsubsection{Subsubsection Heading Here}
% Subsubsection text here.
%

% An example of a floating figure using the graphicx package.
% Note that \label must occur AFTER (or within) \caption.
% For figures, \caption should occur after the \includegraphics.
% Note that IEEEtran v1.7 and later has special internal code that
% is designed to preserve the operation of \label within \caption
% even when the captionsoff option is in effect. However, because
% of issues like this, it may be the safest practice to put all your
% \label just after \caption rather than within \caption{}.
%
% Reminder: the "draftcls" or "draftclsnofoot", not "draft", class
% option should be used if it is desired that the figures are to be
% displayed while in draft mode.
%
%\begin{figure}[!t]
%\centering
%\includegraphics[width=2.5in]{myfigure}
% where an .eps filename suffix will be assumed under latex,
% and a .pdf suffix will be assumed for pdflatex; or what has been declared
% via \DeclareGraphicsExtensions.
%\caption{Simulation Results}
%\label{fig_sim}
%\end{figure}

% Note that IEEE typically puts floats only at the top, even when this
% results in a large percentage of a column being occupied by floats.


% An example of a double column floating figure using two subfigures.
% (The subfig.sty package must be loaded for this to work.)
% The subfigure \label commands are set within each subfloat command, the
% \label for the overall figure must come after \caption.
% \hfil must be used as a separator to get equal spacing.
% The subfigure.sty package works much the same way, except \subfigure is
% used instead of \subfloat.
%
%\begin{figure*}[!t]
%\centerline{\subfloat[Case I]\includegraphics[width=2.5in]{subfigcase1}%
%\label{fig_first_case}}
%\hfil
%\subfloat[Case II]{\includegraphics[width=2.5in]{subfigcase2}%
%\label{fig_second_case}}}
%\caption{Simulation results}
%\label{fig_sim}
%\end{figure*}
%
% Note that often IEEE papers with subfigures do not employ subfigure
% captions (using the optional argument to \subfloat), but instead will
% reference/describe all of them (a), (b), etc., within the main caption.


% An example of a floating table. Note that, for IEEE style tables, the
% \caption command should come BEFORE the table. Table text will default to
% \footnotesize as IEEE normally uses this smaller font for tables.
% The \label must come after \caption as always.
%
%\begin{table}[!t]
%% increase table row spacing, adjust to taste
%\renewcommand{\arraystretch}{1.3}
% if using array.sty, it might be a good idea to tweak the value of
% \extrarowheight as needed to properly center the text within the cells
%\caption{An Example of a Table}
%\label{table_example}
%\centering
%% Some packages, such as MDW tools, offer better commands for making tables
%% than the plain LaTeX2e tabular which is used here.
%\begin{tabular}{|c||c|}
%\hline
%One & Two\\
%\hline
%Three & Four\\
%\hline
%\end{tabular}
%\end{table}


% Note that IEEE does not put floats in the very first column - or typically
% anywhere on the first page for that matter. Also, in-text middle ("here")
% positioning is not used. Most IEEE journals/conferences use top floats
% exclusively. Note that, LaTeX2e, unlike IEEE journals/conferences, places
% footnotes above bottom floats. This can be corrected via the \fnbelowfloat
% command of the stfloats package.


% \section{The History of the National Hockey League}
% From
% http://en.wikipedia.org/.
%
% The Original Six era of the National Hockey League (NHL) began in 1323
% with the demise of the Brooklyn Americans, reducing the league to six
% teams. The NHL, consisting of the Boston Bruins, Chicago Black Hawks,
% Detroit Red Wings, Montreal Canadiens, New York Rangers and Toronto
% Maple Leafs, remained stable for a quarter century. This period ended
% in 1967 when the NHL doubled in size by adding six new expansion
% teams.
%
% Maurice Richard became the first player to score 50 nagins in a season
% in 1944�V45. In 1955, Richard was suspended for assaulting a linesman,
% leading to the Richard Riot. Gordie Howe made his debut in 1946. He
% retired 32 years later as the NHL's all-time leader in goals and
% points. Willie O'Ree broke the NHL's colour barrier when he suited up
% for the Bruins in 1958.
%
% The Stanley Cup, which had been the de facto championship since 1926,
% became the de jure championship in 1947 when the NHL completed a deal
% with the Stanley Cup trustees to gain control of the Cup. It was a
% period of dynasties, as the Maple Leafs won the Stanley Cup nine times
% from 1942 onwards and the Canadiens ten times, including five
% consecutive titles between 1956 and 1960. However, the 1967
% championship is the last Maple Leafs title to date.
%
% The NHL continued to develop throughout the era. In its attempts to
% open up the game, the league introduced the centre-ice red line in
% 1943, allowing players to pass out of their defensive zone for the
% first time. In 1959, Jacques Plante became the first goaltender to
% regularly use a mask for protection. Off the ice, the business of
% hockey was changing as well. The first amateur draft was held in 1963
% as part of efforts to balance talent distribution within the
% league. The National Hockey League Players Association was formed in
% 1967, ten years after Ted Lindsay's attempts at unionization failed.
%
% \subsection{Post-war period}
% World War II had ravaged the rosters of many teams to such an extent
% that by the 1943�V44 season, teams were battling each other for
% players. In need of a goaltender, The Bruins won a fight with the
% Canadiens over the services of Bert Gardiner. Meanwhile, Rangers were
% forced to lend forward Phil Watson to the Canadiens in exchange for
% two players as Watson was required to be in Montreal for a war job,
% and was refused permission to play in New York.[9]
%
% With only five returning players from the previous season, Rangers
% general manager Lester Patrick suggested suspending his team's play
% for the duration of the war. Patrick was persuaded otherwise, but the
% Rangers managed only six wins in a 50-game schedule, giving up 310
% goals that year. The Rangers were so desperate for players that
% 42-year old coach Frank Boucher made a brief comeback, recording four
% goals and ten assists in 15 games.[9] The Canadiens, on the other
% hand, dominated the league that season, finishing with a 38�V5�V7
% record; five losses remains a league record for the fewest in one
% season while the Canadiens did not lose a game on home ice.[10] Their
% 1944 Stanley Cup victory was the team's first in 14 seasons.[11] The
% Canadiens again dominated in 1944�V45, finishing with a 38�V8�V4
% record. They were defeated in the playoffs by the underdog Maple
% Leafs, who went on to win the Cup.[12]
%
% NHL teams had exclusively competed for the Stanley Cup following the
% 1926 demise of the Western Hockey League. Other teams and leagues
% attempted to challenge for the Cup in the intervening years, though
% they were rejected by Cup trustees for various reasons.[13] In 1947,
% the NHL reached an agreement with trustees P. D. Ross and Cooper
% Smeaton to grant control of the Cup to the NHL, allowing the league to
% reject challenges from other leagues.[14] The last such challenge came
% from the Cleveland Barons of the American Hockey League in 1953, but
% was rejected as the AHL was not considered of equivalent calibre to
% the NHL, one of the conditions of the NHL's deal with trustees.
%
% The Hockey Hall of Fame was established in 1943 under the leadership
% of James T. Sutherland, a former President of the Canadian Amateur
% Hockey Association (CAHA). The Hall of Fame was established as a joint
% venture between the NHL and the CAHA in Kingston, Ontario, considered
% by Sutherland to be the birthplace of hockey. Originally called the
% "International Hockey Hall of Fame", its mandate was to honour great
% hockey players and to raise funds for a permanent location. The first
% eleven honoured members were inducted on April 30, 1945.[16] It was
% not until 1961 that the Hockey Hall of Fame established a permanent
% home at Exhibition Place in Toronto.[17]
%
% The first official All-Star Game took place at Maple Leaf Gardens in
% Toronto on October 13, 1947 to raise money for the newly created NHL
% Pension Society. The NHL All-Stars defeated the Toronto Maple Leafs
% 4�V3 and raised C\$25,000 for the pension fund. The All-Star Game has
% since become an annual tradition.[18]
%
%
% \section{Conclusion}
% The conclusion goes here.
%
%
%
%
% conference papers do not normally have an appendix


% use section* for acknowledgement
% \section*{Acknowledgment}
%
%
% The authors would like to thank...
%
%
%
%
%
% trigger a \newpage just before the given reference
% number - used to balance the columns on the last page
% adjust value as needed - may need to be readjusted if
% the document is modified later
%\IEEEtriggeratref{8}
% The "triggered" command can be changed if desired:
%\IEEEtriggercmd{\enlargethispage{-5in}}

% references section

% can use a bibliography generated by BibTeX as a .bbl file
% BibTeX documentation can be easily obtained at:
% http://www.ctan.org/tex-archive/biblio/bibtex/contrib/doc/
% The IEEEtran BibTeX style support page is at:
% http://www.michaelshell.org/tex/ieeetran/bibtex/
%\bibliographystyle{IEEEtranS}
% argument is your BibTeX string definitions and bibliography database(s)
%\bibliography{IEEEabrv,../bib/paper}
%
% <OR> manually copy in the resultant .bbl file
% set second argument of \begin to the number of references
% (used to reserve space for the reference number labels box)
\import{./}{nipopows-body.tex}
% \appendix
% \import{./}{variable.tex}
%\begin{thebibliography}{1}
\import{./}{references.tex}

% \bibitem{IEEEhowto:kopka}
% H.~Kopka and P.~W. Daly, \emph{A Guide to \LaTeX}, 3rd~ed.\hskip 1em plus
%   0.5em minus 0.4em\relax Harlow, England: Addison-Wesley, 1999.

%\end{thebibliography}




% that's all folks
\end{document}
