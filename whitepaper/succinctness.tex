\section{Succinctness}
\label{sec.succinctness}

We analyse the patched scheme we saw in
Algorithm~\ref{alg.nipopow-prover-patched}. We will illustrate why our
construction is succinct in the honest setting. For techniques to make the
construction succinct in broader adversarial settings, consult the
\ifhasappendix
appendix.
\else
full version of this paper.
\fi

We first observe that full succinctness in the standard honest majority model is
impossible to prove for our construction. To see why, recall that an adversary
with sufficiently large mining power can significantly harm superquality as
described in Section~\ref{subsec.superquality-attack}. By reducing superquality
for a sufficiently low level $\mu$, the adversary can cause the honest prover to
digress in their proofs from high-level superchains down to low-level
superchains, causing a linear proof to be produced.

For instance, if superquality is harmed at $\mu = 3$, the prover will
need to digress down to level $\mu = 2$ and include the whole $2$-superchain,
which, in expectation, will be of size $|\chain|/2$.

Having established security in the general case of the standard honest majority
model, we now concentrate our succinctness claims to the particular
``optimistic'' case where the adversary does not use their (minority)
computational power or network power.

\begin{definition}[Optimistic execution]
  We will call an execution \emph{optimistic} if the adversary has $q = 0$
  random oracle queries and the messages diffused by honest parties are
  delivered in random (and not adversarial) order.
\end{definition}

In this setting, the superquality of the chain must be the same as a fully
honestly-generated chain generated with no network adversary. Last, for now, we
will not allow the adversary to produce any proofs; that is, all proofs consumed
by the verifier are honestly-generated. This succinctness assumption is akin
to~\cite{KLS}.

\begin{restatable}[Number of levels]{theorem}{restateThmFewLevels}
    \label{thm.few-levels}
    In any execution, let $S$ denote the set of all blocks produced honestly or
    adversarially. The number of superblock levels which have at least $m$
    blocks are at most $\log(|S|)$, with overwhelming probability in $m$.
\end{restatable}
\ifonecolumn
\import{./}{proofs/fewlevels.tex}
\fi

The above theorem establishes that the number of superchains is small. What
remains to be shown is that few blocks will be included at each superchain
level.

\begin{restatable}[Large upchain expansion]{theorem}{restateThmLargeExpansion}
    \label{thm.large-expansion}
    Consider an optimistic execution and let $\chain$ be an honestly adopted
    chain and let $\chain' = \chain\upchain^{\mu - 1}[i:i + \ell]$ with $\ell
    \geq 4m$. Then $|\chain'\upchain^\mu| \geq m$ with overwhelming probability
    in $m$.
\end{restatable}
\ifonecolumn
\import{./}{proofs/largeexpansion.tex}
\fi

\begin{restatable}[Small downchain support]{lemma}{restateThmSmallSupport}
    \label{lem.small-support}
    Consider an optimistic execution and let $\chain$ be an honestly adopted
    chain and $\chain' = \chain\upchain^\mu[i:i + m]$. Then
    $|\chain'\downchain\upchain^{\mu - 1}| \leq 4m$ with overwhelming
    probability in $m$.
\end{restatable}
\ifonecolumn
\import{./}{proofs/smallsupport.tex}
\fi

This last lemma establishes the fact that the support of blocks needed under
the $m$-sized chain suffix to move from one level to the one below is small.
Based on this, we can establish our theorem on succinctness:

\begin{restatable}[Optimistic succinctness]{theorem}{restateThmSuccinctness}
    \label{thm.succinctness}
    In an optimistic execution, Non-Interactive Proofs of Proof-of-Work produced
    by honest provers are succinct with the number of blocks bounded by $4m
    \log(|\chain|)$, with overwhelming probability in $m$.
\end{restatable}
\ifonecolumn
\import{./}{proofs/succinctness.tex}
\fi

In the above theorem, note the linear dependency between the round $r$ that a
proof is generated and the length $|\chain|$ of the chain of each honest prover.
