\section{Security}
\label{sec:security}

Based on the attack explored above, it is now easy to see that our construction
can be patched in a straightforward manner to achieve security. In particular,
since the manner in which the adversary was able to subvert the prover was by
the violation of \emph{goodness}, we can mandate that the prover only tries to
use succinct proofs to prove claims about chains that are
\emph{good at every level}. In case goodness is violated, the prover simply
falls back to providing the whole chain. This allows us to argue that the
construction is secure by distinguishing two cases. In case goodness is
violated, the honest prover will fall back to providing the whole chain, in
which case security will be reduced to the security of the standard
blockchain protocol choosing the longest $0$-chain. In case goodness is not
violated, we will argue that the adversary is unable to win in these
comparisons.

The previous construction was designed to prevent Bahack-style
attacks~\cite{bahack}, where the adversary constructs ``lucky'' high-difficulty
superblocks without filling in the underlying proof-of-work in the lower levels.
We now patch our protocol which, while retaining this highlevel approach, adds a
defence against the double-spending attack of Section~\ref{sec:attack}. The
attack is neutralized since our verifier is more permissive, allowing the prover
to construct a proof that takes superquality ``goodness'' into account when
comparing forks. The modified construction is shown in
Algorithm~\ref{alg.nipopow-prover-patched}. The algorithm has been modified to
check the current portion of the subchain $\alpha$ for \emph{goodness} prior to
moving to the lower superchain level. If goodness is indeed maintained at the
current level $\mu$, the prover only tries to cover the span of the last $m$
blocks of level $\mu$ at level $\mu - 1$, as seen in
Line~\ref{alg.nipopow-prover-patched.good}. Otherwise, if goodness is violated
at the part of the subchain $\alpha$ at level $\mu$, then the prover completely
ignores level $\mu$ and tries to use the lower level $\mu - 1$ to cover the
whole span of $\alpha$.

\import{./}{algorithms/alg.nipopow-prover-patched.tex}

Only the concrete prover needs to be modified. The verifier and $\leq_m$
operator remain as defined previously.

To aid intuition, we will first give a sketch of the proof before giving the
full technical proof.

\begin{restatable}[Security]{theorem}{restateThmSecurity}
    \label{thm.security}
    Assuming honest majority, the non-interactive proofs-of-proof-of-work
    construction for computable $k$-stable monotonic suffix-sensitive predicates
    is secure with overwhelming probability in $\kappa$.
\end{restatable}
\begin{proof}[Sketch]
Suppose an adversary produces a proof $\pi_\mathcal{A}$ and an honest
party produces a proof $\pi_B$ such that the two proofs cause the predicate $Q$
to evaluate to different values, while at the same time all honest parties have
agreed that the correct value is the one obtained by $\pi_B$. Because of
bitcoin's security, $\mathcal{A}$ will be unable to make these claims for an
actual underlying 0-level chain.

We now argue that the operator $\leq_m$ will
signal in favour of the honest parties.
Suppose $b$ is the LCA block between $\pi_\mathcal{A}$ and $\pi_B$. If the chain
forks at $b$, there can be no more adversarial blocks after $b$ than honest
blocks after $b$, provided there are at least $k$ honest blocks (due to the
Common Prefix property). We will now argue that, further, there can be no more
disjoint $\mu_\mathcal{A}$-level superblocks than honest $\mu_B$-level
superblocks after $b$.

To see this, let $b$ be an honest block generated at some round $r_1$ and let
the honest proof have been generated at some round $r_3$. Then take the sequence
of consecutive rounds $S = (r_1, \cdots, r_3)$. Because the verifier requires at
least $m$ blocks from each of the provers, the adversary must have $m$
$\mu_\mathcal{A}$-superblocks in $\pi_\mathcal{A}\{b:\}$ which are not in
$\pi_B\{b:\}$. Therefore, using a negative binomial tail bound argument, we see
that $|S|$ must be long; intuitively, it takes a long time to produce a lot of
blocks $|\pi_\mathcal{A}\{b:\}|$. Given that $|S|$ is long and that the honest
parties have more mining power, they must have been able to produce a longer
$\pi_B\{b:\}$ argument (of course, this comparison will have the superchain
lengths weighted by $2^{\mu_\mathcal{A}}, 2^{\mu_B}$ respectively). To prove
this, we use a binomial tail bound argument; intuitively, given a long time
$|S|$, a lot of $\mu_B$-superblocks $|\pi_B\{b:\}|$ will have been honestly
produced.

We therefore have a fixed value for the length of the adversarial argument, a
negative binomial random variable for the number of rounds, and a binomial
random variable for the length of the honest argument. By taking the
expectations of the above random variables and applying a Chernoff bound, we see
that the actual values will be close to their means with overwhelming
probability, completing the proof.
\Qed
\end{proof}

We formalize the above proof sketch in the appendix.
