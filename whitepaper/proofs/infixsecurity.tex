\begin{proof}
Assume a typical execution. It suffices to show that the verifier will output
the same value $Q(\chain)$ as some honest prover. Assume honest prover $B$ has
adopted a chain $\chain$ with $Q(\chain) = v$ and has provided proof
$\pi_B$. By Theorem~\ref{thm.security} and because the evaluation of $\tilde\pi$
is identical in the suffix-sensitive and in the infix-sensitive case, we deduce
that $b = \tilde\pi[-1]$ will be an honestly adopted block. Furthermore, due to
the Common Prefix property~ \cite{backbone}, $b$ will belong to all honest
parties' chains and in the same position, as it is buried under $|\tilde\chi| =
k$ blocks.

Because $Q$ is infix-sensitive, it will be defined using a witness predicate
$D$. Because $Q$ is stable, we will have $\exists \chain' \subseteq
\chain[:-k]: D(\chain')$. But $\chain' \subseteq \pi_B$. Let $S =
\textsf{ancestors}(b)$ be the ancestors evaluated by the verifier. As $\chain'
\subseteq S$, therefore $Q(\chain') = Q(S) = v$.
\Qed
\end{proof}
