\section{Gradual Deployment Paths}

\label{sec.forks}
Our construction requires an upgrade to the consensus layer. We envision that
new cryptocurrencies will adopt our construction in order to support efficient
light clients. However, existing cryptocurrencies could also benefit by adopting
our construction as an upgrade. In this section we outline several possible
upgrade paths. We also contribute a novel upgrade approach, a ``velvet fork,''
which allows for gradual deployment without harming unupgraded miners.

\subsection{Hard Forks and Soft Forks}
The obvious way to upgrade a cryptocoin to support our protocol is a hard fork:
the block header is modified to include the interlink structure, and the
validation rules modified to require that new blocks (after a ``flag day'')
contain a correctly-formed interlink hash.

Hard forks are not ``forward compatible,'' and are best avoided. A soft fork
construction requires including the interlink not in the block
header, but in the coinbase transaction. It is enough to only store a hash of
the interlink structure. The only requirement for the NIPoPoWs to work is that the
PoW commits to all the pointers within interlink so that the adversary cannot
cause a chain reorg. If we take that route, then each NIPoPoW will be required to
present not only the block header, but also a proof-of-inclusion path within the
Merkle tree of transactions proving that the coinbase transaction is indeed part
of the block. Once that is established, the coinbase data can be presented, and
the verifier will thereby know that the hash of the interlink data structure is
correct. Given that in Bitcoin implementation there is a block size limit,
observe that including such proofs-of-inclusion will only increase the NIPoPoW
sizes by a constant factor per block, allowing for the communication complexity
to remain polylogarithmic.

\subsection{Velvet Forks}
We now describe a novel upgrade path that avoids the need for a fork at all. The
key idea is that clients can make use of our scheme, even if only some blocks in
the blockchain include the interlink structure. Given that intuitively the
changes we will propose require no rule modifications to the consensus layer, we call this
technique a \textit{velvet fork}
\footnote{After the first manuscript of the present paper was published on the
\textit{eprint} archive, velvet forks were subsequently explored in detail in
the excellent follow-up work by Zamyatin et. al.~\cite{velvet}}.

We require upgraded miners to include the interlink data structure in the form
of a new Merkle tree root hash in their coinbase data, similar to a soft fork.
An unupgraded miner will ignore this data as comments. We further require the
upgraded miners to accept all previously accepted blocks, regardless of whether
they have included the interlink data structure or not. Even if the interlink
data structure is included and contains invalid data, we require the upgraded
miners to accept their containing blocks. Malformed interlink data could be
simply of the wrong format, or the pointers could be pointing to superblocks of
incorrect levels. Furthermore, the pointers could be pointing to superblocks of
the correct level, but not to the most recent block. By requiring upgraded
miners to accept all such blocks, we do not modify the set of accepted blocks.
Therefore, the upgrade is simply a ``recommendation'' for miners and not an
actual change in the consensus rules. The velvet fork has the effect that blocks
produced by either upgraded or unupgraded clients are valid for either. Hence,
the blockchain is never forked. Only the codebase is upgraded, and the  data on
the blockchain is interpreted differently.

The reason this can work is because provers and verifiers of our protocol can
check the validity of the claims of miners who make false interlink chain
claims. An upgraded prover can check whether a block contains correct interlink
data and use it. If a block does not contain correct interlink data, the prover
can opt not to use those pointers in their proofs. The Verifier verifies all
claims of the prover, so adversarial miners cannot cause harm by including
invalid data. The one thing the Verifier cannot verify in terms of interlink
claims is whether the claimed superblock of a given level is the most recent
previous superblock of that level. However, an adversarial prover cannot make
use of that to construct winning proofs, as they are only able to present
shorter chains in that case.
The
honest prover can simply ignore such pointers as if they were not included at
all.

The velvet prover works as usual, but additionally maintains a \textit{realLink}
data structure, which stores the \textit{correct} interlink for each block.
Whenever a new winning chain is received from the network, the prover checks it
for blocks that it hasn't seen before. For those blocks, it maintains its own
realLink data structure which it updates accordingly to make sure it is correct
regardless of what the interlink data structure of the received block claims.

The velvet $\chain\upchain$ operator
shown in Algorithm~\ref{alg.nipopow-velvet-innerchain}
is implemented identically as before, except that instead of following the
interlink pointer blindly it now calls the helper function \textit{followUp},
shown in Algorithm~\ref{alg.nipopow-velvet-follow}. It accepts block $B$ and
level $\mu$ and creates a connection from $B$ back to the most recent preceding
$\mu$-superblock, by following the interlink pointer if it is correct.
Otherwise, it follows the previd link which is available in all blocks, and
tries to follow the interlink pointer again from there. Finally, the velvet
prover
%shown in
%Algorithm~\ref{alg.nipopow-velvet-prover}
simply applies the velvet $\chain\upchain$ operator and includes the auxiliary
connecting nodes within the final proof. No changes in the verifier are needed; note that  in the case of infix proofs the $\mathrm{index}$ of the block is used by the verifier; if this information is not provided by the underlying blockchain headers, the index should be included in the interlink structure.

%\import{./}{algorithms/alg.nipopow-velvet-innerchain.tex}
%\import{./}{algorithms/alg.nipopow-velvet-follow.tex}
%\import{./}{algorithms/alg.nipopow-velvet-prover.tex}

Velvet NIPoPoWs preserve security. Additionally, if a constant minority of
miners has upgraded their nodes, then succinctness is also preserved
as shown below.
%as there
%is only a constant factor penalty as proven in the following theorem.

\begin{restatable}{theorem}{restateThmVelvetSuccinctness}
    \label{thm.velvet-succinctness}
    Velvet non-interactive proofs-of-proof-of-work on honest chains by honest
    provers remain succinct as long as a constant percentage $g$ of miners has
    upgraded, with overwhelming probability.
\end{restatable}

\ifonecolumn
\import{./}{proofs/velvetsuccinctness.tex}
\fi

The reason we were able to upgrade using a velvet fork was because the changes
we made were helpful but verifiable by those looking at the chain. We would not
have been able to pull off this upgrade without modifications to the consensus
layer in the sense that the interlink data structure could not have been
maintained somewhere independently of the blockchain: It is critical that the
proof-of-work commits to the interlink data structure. Interestingly, the
interlink data structure does not need to be part of coinbase and can be
produced and included in regular transactions by users (such as OP\_RETURN
transactions). Thus, the miners can be completely oblivious to it, while users
and provers make use of the feature. Interested users regularly create
transactions containing the most recent interlink pointers so that they are
included in the next block. If the transaction makes it to the next block, it
can be used by the prover who keeps track of these. Otherwise, if it becomes
part of a subsequent block, in which case some of the pointers it contains are
invalid, it can be ignored or only partially used. The complete algorithms for
velvet forks are included in the appendix.

\import{./}{algorithms/alg.nipopow-velvet-innerchain.tex}

The necessary changes needed in the various construction algorithms in order to
support a velvet fork are shown in
Algorithm~\ref{alg.nipopow-velvet-innerchain},
Algorithm~\ref{alg.nipopow-velvet-follow}, and
Algorithm~\ref{alg.nipopow-velvet-prover}.
Algorithm~\ref{alg.nipopow-velvet-innerchain} defines the innerChain method,
which replaces the upchain/downchain abstraction that can be easily used in a
fully upgraded chain.

The following theorem establishes the security of NIPoPoWs in velvet forks.

\restateThmVelvetSuccinctness*
\import{./}{proofs/velvetsuccinctness.tex}

\import{./}{algorithms/alg.nipopow-velvet-follow.tex}
\import{./}{algorithms/alg.nipopow-velvet-prover.tex}

\noindent\textbf{Supporting clients with different beliefs.}
The interlink format does not depend on parameters $m, k$. Therefore, it is not
necessary to agree on a particular value of these parameters. Instead, the
choice of $m$ and $k$ can be a user-configurable parameter to clients.
Clients would send a particular $m$ and $k$ as part of their requirement to
the prover.
