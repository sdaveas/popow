\appendix
\section*{Appendix}

Our Appendix is structured as follows.
In Section~\ref{sec:forks}, we illustrate
gradual deployment paths. One of our techniques allows adoption of our scheme
without requiring miner consensus. We term this technique a \emph{velvet fork}
in contrast to the classical \emph{soft} and \emph{hard forks} which require
approval by a majority of miners. This technique is a novel contribution and may
be of independent interest for other blockchain protocols.
Section~\ref{sec:variable} gives an intuition about creating a construction for
variable difficulty NIPoPoWs by modifying the construction presented in this
paper.
Section~\ref{sec:app-quality} gives the lemmas and associated proofs showing how
superchains are distributed. This provides the necessary tools to show that the
construction is optimistically succinct. Section~\ref{sec:attack-full} contains
the full formal proof that our attack against our strawman construction succeeds with overwhelming probability, given the correct
strategy and protocol parameters. Section~\ref{sec:security-full} gives a formal
proof of our security claims through a computational reduction. Section~
\ref{sec:app-succinctness} includes the remaining proofs that were omitted from
the body of the paper with the goal of proving optimistic succinctness, a
central result of our paper. It also proves succinctness in more adversarial
settings. We conclude with
Section~\ref{sec:app-ico} which includes experimental data of our Solidity
implementation for the ICO application.

\import{./}{upgrade.tex}
\import{./}{variable.tex}
\import{./}{app-quality.tex}
\import{./}{attack-formal.tex}
\import{./}{security-formal.tex}
\import{./}{app-succinctness.tex}
\import{./}{app-ico.tex}
\import{./}{acknowledgements.tex}
